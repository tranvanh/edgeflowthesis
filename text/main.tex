% arara: xelatex
% arara: xelatex
% arara: xelatex


% options:
% thesis=B bachelor's thesis
% thesis=M master's thesis
% czech thesis in Czech language
% english thesis in English language
% hidelinks remove colour boxes around hyperlinks

\documentclass[thesis=M,english]{FITthesis}[2023/2/2]

%\usepackage[utf8]{inputenc} % LaTeX source encoded as UTF-8
% \usepackage[latin2]{inputenc} % LaTeX source encoded as ISO-8859-2
% \usepackage[cp1250]{inputenc} % LaTeX source encoded as Windows-1250

% \usepackage{subfig} %subfigures
% \usepackage{amsmath} %advanced maths
% \usepackage{amssymb} %additional math symbols

\usepackage{dirtree} %directory tree visualisation
\usepackage{xcolor}
\usepackage{xspace}
\usepackage{amsmath}
\usepackage{amssymb}
\usepackage{graphicx}
\usepackage{subfig}
\usepackage{float}
\usepackage{nameref}
\usepackage{algorithm}
\usepackage{algpseudocode}
\usepackage{pdfpages}

\graphicspath{ {./assets/} }
% % list of acronyms
% \usepackage[acronym,nonumberlist,toc,numberedsection=autolabel]{glossaries}
% \iflanguage{czech}{\renewcommand*{\acronymname}{Seznam pou{\v z}it{\' y}ch zkratek}}{}
% \makeglossaries

% % % % % % % % % % % % % % % % % % % % % % % % % % % % % % 
% EDIT THIS
% % % % % % % % % % % % % % % % % % % % % % % % % % % % % % 

\department{Department of Applied Mathematics}
\title{Material Picker: tool for detecting material properties}
\authorGN{Viet Anh} %author's given name/names
\authorFN{Tran} %author's surname
\author{Viet Anh Tran} %author's name without academic degrees
\authorWithDegrees{Bc. Viet Anh Tran} %author's name with academic degrees
\supervisor{Ing. Tomáš Nováček}

\acknowledgements{me, myself and I}


\abstractEN{TODO
}

\abstractCS{TODO CZ}
\placeForDeclarationOfAuthenticity{Prague}
\keywordsCS{TODO}
\keywordsEN{TODO}
\declarationOfAuthenticityOption{1} %select as appropriate, according to the desired license (integer 1-6)
% \website{http://site.example/thesis} %optional thesis URL


%========================================================================================================
\begin{document}
%========================================================================================================
	%========================================================================================================
	\setsecnumdepth{part}
	\chapter{Introduction}\label{ch:introduction}
	Computer graphics and 3D modeling is a fundamental field in our constantly increasing digital world. Its application in engineering, science, art, product visualization, and most notably, entertainment has pushed this field with a constant demand for better visual quality and performance optimization.

3D models create the very core of any 3D visualization, bringing objects to computer-generated 3D space. The 3D model defines the shape of an object and can be further improved to closely represent real-world objects by applying textures, lighting, and many more techniques. Models are typically created in 3D modeling software by an artist or by 3D scanning. 3D scanning can often bring too many details in terms of polygon numbers. A shape that could be perfectly recognizable by ten polygons would now be described by thousands. The side effect of things is big hardware requirements.

Lowering the number of polygons is a usual practice done either manually or by many developed tools over the years. These tools do not consider any edge flow, a critical feature particularly important for character modeling and animation. Edge flow allows smooth deformation. In terms of animation, we can look at muscle movement, optimal edge avoids wrinkled defects on the character model, and the skin would stretch and contract in a way as we are used to seeing in the real world.

This thesis aims to introduce the final step in polygon reduction workflow when using tools. We present a neural network taking a 3D model and outputting an improved model in terms of edge flow quality, approximating similar polygon quantity.

\begin{description}

    \item In Chapter \nameref{ch:neural_network}, we explore basics of neural networks, introduce its terminology and several commonly used network architectures.
    
    \item In Chapter \nameref{ch:implementation}, we describe used methods and key implementation points of our work.
    
    \item In Chapter \nameref{ch:experiments}

    \item In \nameref{ch:conclusion}, we evaluate the results of the work and suggest possible improvements for future works.
\end{description}\newpage\cleardoublepage
	%========================================================================================================
	\setsecnumdepth{all}
	\chapter{Neural Networks}\label{ch:neural_network}
	An artificial neural network (ANN), first introduced by Warren McCulloch and Walter Pitts in "A logical calculus of the ideas immanent in nervous activity" published in 1943 \cite{mcculloch1943logical}, is a mathematical model based on biological neural networks. It 
carries the ability to learn and correct errors from previous experience \cite{designimplentationcc}, \cite{bengio2017deep}.

The ANN has gained popularity in recent years with still increasing advancements in technology and availability of training data. ANN now becomes a default solutions for complex tasks previously thought to be unsolvable by computers \cite{neural2016krishtopa}.

This chapter will briefly introduce different types of neural units and their activation functions, along with some commonly used network architectures.

\setsecnumdepth{all}
\section{Artificial Neuron}
Artificial neurons are units of ANN, mimicking behavior of biological neurons. Just as biological neurons, it can receive as well as pass information to other neurons.

\setsecnumdepth{all}
\subsection{Perceptron}
\input{chapters/neural_network/artificial_neuron/perceptron.tex}
%=======================================================================================================================
\subsection{Sigmoid Neuron}
\input{chapters/neural_network/artificial_neuron/sigmoid_neuron.tex}
%=======================================================================================================================
\subsection{Activation Function}
An artificial neuron's activation function defines neuron's output value for given inputs, commonly being ${f: \mathbb{R} \rightarrow \mathbb{R}}$ \cite{leskovec2020mining}. An important trait of many activation functions is their differentiability, allowing them to be used for \textit{Backpropagation}, ANN training algorithm. The activation function needs to have a derivative that does not saturate when headed towards 0 or explode when headed towards inf \cite{matous}.

For such reasons, step function or any linear function are unsuitable for ANN.
% Sigmoid Function ==========================================================================================================
\setsecnumdepth{all}
\subsubsection{Sigmoid Function}
The sigmoid function is often used in ANN as an alternative to the step function. A popular choice of the sigmoid function is a \textit{logistic sigmoid}, output value ranging between 0 and 1.

\begin{equation}
    {\sigma(\alpha) = \frac{1}{1 + e^{-\alpha}} = \frac{e^x}{1 + e^{x}}}
\end{equation}


% \begin{figure}[h]
%   \centering
%     \includegraphics[width=7cm]{sigmoid}
%   \caption{Sigmoid function}
%   \label{fig:sigmoid}
% \end{figure}


One of the reasons for its popularity is the simplicity of its derivative calculation:

\begin{equation}
    {\frac{d}{dx}\sigma(\alpha) = \frac{e^x}{(1 + e^{x})^2} = \sigma(x)(1-\sigma(x))}
\end{equation}


Its disadvantages is the \textit{vanishing gradient}. A problem where if given a very high or very low input values, the prediction would stay almost the same. Possibly resulting in training complications or performance issues \cite{7typesactivationfunctions}, \cite{matous}.

% Hyperbolic Tangent ==========================================================================================================

\subsubsection{Hyperbolic Tangent}

Hyperbolic tangent is similar to logistic sigmoid function with a key difference in its output, ranging between -1 and 1.

\begin{equation}
    {tanh(x) = \frac{e^x - e^{-x}}{e^x + e^{-x}}}
\end{equation}


\begin{figure}[h]
    \centering
    \includegraphics[width=8cm]{tangent}
    \caption{Hyperbolic tangent \cite{matous}}
    \label{fig:hyperbolictangent}
\end{figure}


It shares the sigmoid's simple calculation of its derivative.

\begin{equation}
    {\frac{d}{dx}\tanh(x) = 1 - \frac{(e^x - e^{-x})^2}{(e^x + e^{-x})^2} = 1 -\tanh^2(x)}
\end{equation}

By being only moved and scaled version of the sigmoid function, hyperbolic tangent shares not only sigmoid's advantages but also its disadvantages \cite{leskovec2020mining}, \cite{matous}.

% Rectified Linear Unit ==========================================================================================================

\subsubsection{Rectified Linear Unit}

The output of the Rectified Linear Unit (ReLU) is defined as:

\begin{equation}
    f(x) = max(0,x)
\begin{cases}
    x, & \text{if $x\ \geq\ 0$}\\
    0, & \text{if $x\ <\ 0$}
\end{cases} 
\end{equation} 

\begin{figure}[h]
    \centering
    \includegraphics[width=7cm]{relu}
    \caption{Rectified Linear Unit \cite{matous}}
    \label{fig:relu}
\end{figure}


ReLU's popularity is mainly due to its computational efficiency \cite{7typesactivationfunctions}. Its disadvantages begin to show themselves once inputs approach zero or to a negative number. Causing the so-called dying ReLu issue, where the network is unable to learn anymore. There are many variations of ReLu to this date, e.g., Leaky ReLU, Parametric ReLU, ELU, ...
% Softmax ==========================================================================================================

\subsubsection{Softmax}

Softmax separates itself from all the previously mentioned functions by its ability to handle multiple input values in the form of a vector $\vec{x} = (x_1,x_2,...,x_n)$ and output for each $x_i$ defined as:

\begin{equation}
    {\sigma(x_i) = \frac{e^x_i}{\sum_{j=1}^{n}e^x_j}}
\end{equation}

Output is normalized probability distribution, ensuring $\sum_{i}\sigma(x_i) = 1$ \cite{lipton2015critical}. It is being used as the last activation function of ANN, normalizing the network's output into $n$ probability groups.

%=======================================================================================================================
%=======================================================================================================================
\section{Types of Neural Networks}
\input{chapters/neural_network/types_of_nn.tex}
%=======================================================================================================================\newpage\cleardoublepage
	%========================================================================================================
	\setsecnumdepth{all}
	\chapter{Polygon Mesh}\label{ch:polygon_mesh}
	
In the following chapter, we will discuss shape representation in computer graphics. Explore some of their properties and what tools can be used to modify them. Some of the usual data structures to represent a geometric shape and position within a 3D space are:

\begin{itemize}
    \item \textbf{Mesh} is a set of vertices, edges, and faces defining a 3D object. It is especially used in computer graphics, and it is a simple way to represent complex 3D shapes. A face is a polygon with a minimum of 3 vertices. The most commonly used is a triangle. If a face is made of 4 vertices, it is called a quad, and more than four is called a general polygon. In our case, when we mention faces, we will have triangles in mind. These faces then form a general surface. 
    \begin{figure}[h]
        \centering
        \includegraphics[width=11cm]{mesh.png}
        \caption{Elements of mesh object \cite{stanford}}
        \label{fig:mesh}
    \end{figure}
    
    A 3D object can have a color. To do so, we assign color values to every vertex of the 3D object. The pixel color of the triangle is determined based on the three vertices by which it is made.
    
    \begin{figure}[h]
        \centering
        \includegraphics[width=4cm]{rabbit_mesh.png}
        \caption{Stanford bunny model made of mesh \cite{stanford}}
        \label{fig:rabbit_mesh}
    \end{figure} 

    \item \textbf{Voxel} objects are in comparison with mesh objects solid. As already said, mesh objects are created from a surface of little triangles, but it is also worth noting that this object is hollow, just like a ballooon. On the other hand, if a model is created from voxels, abbreviation from volumetric pixels, it means that the object was created from cubes and the object itself is solid, its inside also holds information. The working scene is a 3D grid, and its data point holds information about opacity, color, and material information is often also stored.
    
    \begin{figure}[h]
        \centering
        \includegraphics[width=4cm]{rabbit_voxel.png}
        \caption{Stanford bunny model made of voxels \cite{lstmcell_img}}
        \label{fig:rabbit_voxel}
    \end{figure}

    Voxels are often used in medicine and terrain representation. Voxel terrain can represent overhangs, caves, arches, and other, which is difficult to represent using heightmaps, which represents only top-layer data, and anything below it would be filled with no option for holes.

    The main disadvantage of voxels is the resolution. If we want to have a highly detailed voxel model, we would have to increase the resolution of the whole scene.

    \item \textbf{Point cloud} is a collection of points plotted in 3D space. Each point contains position coordinates, color values, and luminance values, determining how bright a point is.
    
    Points are usually acquired by a 3D scanner or photogrammetry software. Scanners work by sending out pulses of light to the object and measuring how long each point takes to reflect back and hit the scanner. These measurements are used to determine the exact positions of points on the object, creating a point cloud. Photogrammetry is a process to create measurements from pictures. It uses photos of an object to triangulate points on the object and plot these points to 3D points, resulting in point clouds.
    
    \begin{figure}[h]
        \centering
        \includegraphics[width=4cm]{rabbit_cloud.png}
        \caption{Stanford bunny model made of point clouds \cite{stnaford}}
        \label{fig:rabbit_cloud}
    \end{figure}
  \end{itemize}

  In terms of the thesis and its goal, we will mainly explore mesh polygons and their edge flow properties while also exploring polygon reduction operations on them.

\section{Edge flow}

Edge flow is a fundamental concept in 3D modeling. The general goal is to ensure that mesh edges follow the curves of an object. A mesh with distinctly different edge flows can represent the same object while preserving the same shape. The key difference and significance of edge flows plays in the world of animation, overall any deformation operation performed on the object. In general, a good edge flow has uniformly distributed points along the 3d model, meaning the length and the area of each primitive is also of similar if not equal sizes. 

\begin{figure}[h]
    \centering
    \includegraphics[width=12cm]{topology.png}
    \caption{A mesh object constructed with 2 different edge flows \cite{decimation}}
    \label{fig:topology_comp}
\end{figure}

An optimal edge flow allows smooth and natural deformations. In the case of figure \ref{fig:topology_comp}, if we want to animate various facial expressions, the left example would have distorted wrinkles. In contrast, the right example would allow for natural mimicry, as we are familiar with in real life. A more expressive example would be in \ref{fig:planar_def}, where we can see a plane of two different topologies having different behavior if bent. The left example has slight bumpy artifacts, while the right preserves smooth surface transition.

\begin{figure}[h]
    \centering
    \includegraphics[width=12cm]{planar_deform.png}
    \caption{Deformation of the same object with different edge flow \cite{topology_animation}}
    \label{fig:planar_def}
\end{figure}

Edge flows are manually crafted by 3D artist and require experience to reach their optimal form. They are either being considered from the beginning while creating the 3D mesh or as a post-correction, where the acquired 3D mesh does not meet the required quality needed for future work. One of the possible reasons for having insufficient quality could be caused by polygon reduction.

\section{Polygon reduction}
Polygon reduction is a common practice for performance and memory optimization as a higher number of polygons may introduce greater details, but they also bring high demand on hardware requirements. A high polygon mesh is usually a byproduct of the 3D scanning process where the resulting 3D mesh can be of enormous memory size. Most importantly, the reduced amount should be considered as the reduction can lose the visual representation of the object and its key features. The general goal is to balance visual quality and performance optimization. Following, we will list some of the most used techniques for polygon reduction and their effect on the resulting edge flow. Edge flow is a fundamental concept in 3D modeling, playing a pivotal role in natural 3D model deformations mostly used in animation. 

\section{Decimation}

Decimation is a commonly used method for quick polygon reduction. It reduces the percentage of vertices and edges uniformly while preserving the overall shape of the reduced object, but it does not handle fine details well. Most 3d modeling software, such as Blender, Maya, 3ds Max, Zbrush, and Cinema4D, support decimation since it is a common technique. The overall control or workflow may differ, but the general concept remains the same.\cite{decimation}

\begin{figure}[h]
    \centering
    \includegraphics[width=12cm]{ShirtComparison.png}
    \caption{Decimation applied on shirt 3D mesh \cite{decimation}}
    \label{fig:shirt_comp}
\end{figure}

As seen in \ref{fig:shirt_comp}, the optimized 3D mesh may be optimized in the number of polygons, but its edge flow suffered greatly. If the shirt was part of any animated object, its part would deform unnaturally in undesired ways.

\section{Retopology}

Retopology is a semi-automatic method requiring high user input. Retopology requires a user to manually outline the required surface by hand while hinting at its edge flow by following the model's edges before the 3D software follows the outlines and draws a polygon. This gives a user high control over its model but is highly demanding on general experience and skill while identifying the edges. From a quality perspective, the resulting model will be suitable for animations and smooth 3D model deformations. Still, from a workflow efficiency standpoint, producing one of these reduced models requires a lot of time.

\begin{figure}[h]
    \centering
    \includegraphics[width=12cm]{retopology.png}
    \caption{Retopology surface manually created lizard 3D mesh \cite{retopology}}
    \label{fig:retopology}
\end{figure}

\section{Quad Remeshing}

Instant remesh

Eh...

Decimation -> automatic reduction
Decimation does not preserve edge flow

* polygon reduction tools
* retopology -> requires manual work Blender
* quad remeshing -> often use to preserve nice deformation -> for animation -> main nemesis
* edge collapse -> manual work

These reduction techniques don't ensure so called edge flow, a fundamental concept in 3D modeling playing a pivotal role in natural 3D model deformation mostly used in animation.\newpage\cleardoublepage
	%========================================================================================================
	\setsecnumdepth{all}
	\chapter{Related Work}\label{ch:related_works}
	The main challenge is that there is currently no way to express an optimality of an edge flow by one number, which is critical for learning methods when we want to compute a loss function. We previously mentioned that an optimal flow has uniformly distributed points, edge lenghts and overall area of a primitive. One way to possibly achieve that is to look at the task at hand from different perspective. Instead of doing a correction of the given 3D mesh, we will try to reconstruct the shape of a given 3D mesh with better topology.

In this chapter we will explore researches using neural networks for 3D mesh reconstruction. 

\section{Retrieve and deform a template}

Retrieve and deform a template is a two step solution. First a most suitable template is picked and then it is deformed to the target object. The input is processed with neural network which classifies which template is best suited, such as BCNet \cite{bcnet}, MultiGarment Net\cite{mgn} or ShapeFlow \cite{shapeflow}. As seen in Figure \ref{fig:bcnet} and \ref{fig:shapeflow}, a nearest-neighbor template is retrieved from the embedded space and then a deformation network is applied.

\begin{figure}[h]
    \centering
    \includegraphics[width=12cm]{bcnet.png}
    \caption{Overview of retrieve and deform shape reconstruction using BCNet \cite{bcnet}}
    \label{fig:bcnet}
\end{figure}

\begin{figure}[h]
    \centering
    \includegraphics[width=12cm]{shapeflow.png}
    \caption{Overview of retrieve and deform shape reconstruction using ShapeFlow \cite{shapeflow}}
    \label{fig:shapeflow}
\end{figure}

\section{Deform a primitive}

Retrieve and deform solution expects a high quality templates and they were also highly specialized in certain category type, which makes the overall input and output then highly specialized. Other method often used for aproximating to target object from a single image would be deforming a primitive. Popular representation in Pixel2Mesh \cite{p2m}, Pixel2Mesh++ \cite{p2mpp}, Neural mesh flow \cite{neuralFlow} was a simple sphere. 

Pixel2Mesh had a 2D CNN extracting features from a single image which is then leveraged by a deformation block, progressively deforming a sphere into the desired 3D model. The cascaded mesh deformation network is a graph-based convolution network, containing three deformation blocks intersected by two graph unpooling layers, which increases the number of vertices. With given face the points were added to the middle of each connecting edges. Connecting these newly added points then forms a new set of primitive faces, while also ensuring even distribution of vertices and their degrees. The improved Pixel2Mesh is extracts features from multiple view images, increasing the reconstruction accurancy. 

\begin{figure}[h]
    \centering
    \includegraphics[width=6cm]{p2m_unpooling.png}
    \caption{Graph unpooling\cite{p2m}}
    \label{fig:p2munpooling}
\end{figure}

\begin{figure}[h]
    \centering
    \includegraphics[width=12cm]{p2m.png}
    \caption{Pixel2Mesh architecture overview \cite{p2m}}
    \label{fig:p2m}
\end{figure}

The Neural mesh flow shares very similar architecture as Pixel2Mesh in using three deform blocks.

Neural Mesh Flow (NFM) as seen in Figure \ref{fig:netflow}, learns to auto-encode 3D shapes. NMF broadly consists of four components. First, the target shape is encoded by uniformly sampling N points from its surface and feeding them to a PointNet \cite{pointnet} encoder to get the global shape embedding. Second, NODE blocks diffeomorphically flow the vertices of template sphere towards target shape conditioned on shape embedding. Third, the instance normalization layer performs non-uniform scaling of NODE output to ease cross-category training. Finally, refinement flows provide gradual improvement in quality.

\begin{figure}[h]
    \centering
    \includegraphics[width=12cm]{netflow.png}
    \caption{Neural Mesh flow architecture overview \cite{neuralFlow}}
    \label{fig:netflow}
\end{figure}




The mesh reconstruction is often done from voxels, point clouds, single images, and multi-view images, respectively. There was previously no real purpose of doing a reconstruction from a mesh as in our case.

universal solution, depends on previous topology quality\newpage\cleardoublepage
	%========================================================================================================
	\setsecnumdepth{all}
	\chapter{Implementation}\label{ch:implementation}
	\input{chapters/implementation.tex}\newpage\cleardoublepage
	%========================================================================================================
	\setsecnumdepth{all}
	\chapter{Experiments}\label{ch:experiments}
	\input{chapters/experiments.tex}\newpage\cleardoublepage
	%========================================================================================================
	\setsecnumdepth{all}
	\chapter{Conclusion}\label{ch:conclusion}
	\input{chapters/conclusion.tex}\newpage\cleardoublepage


\bibliographystyle{iso690}
\bibliography{mybibliographyfile}

\setsecnumdepth{all}
\appendix

\chapter{Acronyms}
% \printglossaries
\begin{description}
	\item[ANN] Artificial Neural Network
	\item[RNN] Recurrent Neural Network
	\item[CNN] Convolutional Neural Network
	\item[LSTM] Long Short-Term Memory
	\item[ReLU] Rectified Linear Unit
\end{description}


\chapter{Contents of enclosed CD}

%change appropriately

\begin{figure}
	\dirtree{%
		.1 README.md\DTcomment{the Markdown file with description}.
		.1 executables\DTcomment{the directory with executables}.
		.2 Dataset\DTcomment{the directory of the original dataset}.
		.2 TrainedModel\DTcomment{the directory of trained model}.
		.2 DataSampler.exe\DTcomment{data sampling application}.
		.2 model\_trainig.py\DTcomment{model training Python script}.
		.2 GestureApp.exe\DTcomment{gesture recognition demo application}.
		.1 src\DTcomment{the directory of source codes}.
		.1 text\DTcomment{the directory of \LaTeX{} source codes of the thesis}.
		.1 environment.yml\DTcomment{configuration file for conda environment}.
		.1 BP\_Viet\_Anh\_Tran\_2021.pdf\DTcomment{the thesis text in PDF format}.
	}
\end{figure}



\end{document}
